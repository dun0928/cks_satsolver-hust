% !TeX program = xelatex
\documentclass[UTF8]{ctexart}
\usepackage[a4paper,margin=2.5cm]{geometry}
\usepackage{fontspec}
\usepackage{hyperref}
\usepackage{graphicx}
\usepackage{booktabs}
\usepackage{tabularx}
\usepackage{listings}
\usepackage{xcolor}
\usepackage{enumitem}
\usepackage{longtable}
\usepackage{fancyhdr}
\usepackage{titlesec}
\usepackage{array}
\usepackage{amsmath,amssymb}

% 字体设置:中文为宋体,西文为 Times New Roman,提供黑体族
\setmainfont{Times New Roman}
\setCJKmainfont{SimSun}
\setCJKfamilyfont{hei}{SimHei}
\newcommand{\heiti}{\CJKfamily{hei}}

% 正文默认:宋体小四(12pt),行距 1.5 倍
\renewcommand{\baselinestretch}{1.5}
\newcommand{\bodyfont}{\songti\zihao{-4}}

% 标题格式(尽量贴近:章 黑体小二 居中;节 黑体四号;小节 黑体小四)
\titleformat{\section}{\centering\heiti\zihao{-2}}{\thesection}{1em}{}
\titleformat{\subsection}{\heiti\zihao{4}}{\thesubsection}{1em}{}
\titleformat{\subsubsection}{\heiti\zihao{-4}}{\thesubsubsection}{1em}{}

% 列表紧凑一些
\setlist{nosep,topsep=4pt,partopsep=0pt,itemsep=2pt,parsep=0pt}

% 代码块
\lstset{
  basicstyle=\ttfamily\small,
  columns=fullflexible,
  breaklines=true,
  frame=single,
  numbers=left,
  numberstyle=\tiny, 
  xleftmargin=2em,
  framexleftmargin=1.5em,
  keywordstyle=\bfseries\color{blue!60!black},
  commentstyle=\itshape\color{gray!70!black},
  stringstyle=\color{green!40!black}
}

% 页眉页脚
\pagestyle{fancy}
\fancyhf{}
\fancyhead[L]{\songti\zihao{-5} 基于 DPLL 的 SAT 求解与数独应用}
\fancyhead[R]{\songti\zihao{-5} 实验报告}
\fancyfoot[C]{\thepage}

\begin{document}
\bodyfont

%==================== 任务书 ====================
\begin{center}
  {\heiti\zihao{-2} 任务书}\\[0.5em]
\end{center}

\noindent
\textbf{设计内容}\\
SAT 问题即命题逻辑公式的可满足性问题(satisfiability problem),是计算机科学与人工智能基本问题,是一个典型的 NP 完全问题,可广泛应用于许多实际问题如硬件设计、安全协议验证等,具有重要理论意义与应用价值。本设计要求基于 DPLL 算法实现一个完备 SAT 求解器,对输入的 CNF 范式算例文件,解析并建立其内部表示;精心设计问题中变元、文字、子句、公式等有效的物理存储结构以及一定的分支变元处理策略,使求解器具有优化的执行性能;对一定规模的算例能有效求解,输出与文件保存求解结果,统计求解时间。\\

\noindent
\textbf{设计要求}
\begin{enumerate}
  \item 输入输出功能:程序执行参数输入,CNF 文件读取,结果输出与保存(15\%)。
  \item 公式解析与验证:建立内部表示,并能逐行打印验证解析正确性(15\%)。
  \item DPLL 过程:基于 DPLL 框架实现 SAT 求解(35\%)。
  \item 时间性能测量:记录 DPLL 执行时间(毫秒)(5\%)。
  \item 程序优化:在数据结构或分支策略等方面优化,并给出优化率 $[(t{-}t_o)/t]{\times}100\%$(15\%)。
  \item SAT 应用:将数独问题归约为 SAT,集成求解并具备简单交互(15\%)。
\end{enumerate}

\noindent
\textbf{参考文献}
\begin{enumerate}
  \item 张健著. 逻辑公式的可满足性判定—方法、工具及应用. 科学出版社,2000
  \item Tanbir Ahmed. An Implementation of the DPLL Algorithm. Concordia University, 2009
  \item 陈稳. 基于 DPLL 的 SAT 算法的研究与应用. 电子科技大学, 2011
  \item Carsten Sinz. Visualizing SAT Instances and Runs of the DPLL Algorithm. JAR (2007) 39:219–243
  \item 360百科:数独游戏 \url{https://baike.so.com/doc/3390505-3569059.html}; Twodoku \url{https://en.grandgames.net/multisudoku/twodoku}
  \item Tjark Weber. A SAT-based Sudoku Solver. LPAR 2005.
  \item In\^es Lynce, Joel Ouaknine. Sudoku as a SAT Problem. AIMATH 2006.
  \item Uwe Pfeiffer et al. A Sudoku-Solver for Large Puzzles using SAT. EPiC 13.
  \item Sudoku Puzzles Generating: from Easy to Evil. \url{http://zhangroup.aporc.org/images/files/Paper_3485.pdf}
  \item 薛源海等. 基于“挖洞”思想的数独游戏生成算法. 数学的实践与认识, 2009
  \item 黄祖贤. 数独游戏的问题生成及求解算法优化. 安徽工业大学学报, 2015
\end{enumerate}

%==================== 正文 ====================
\section{引言}
\subsection{课题背景与意义}
SAT(Satisfiability)问题是判定给定命题公式是否存在使其为真的赋值的判定问题。自 Cook--Levin 定理表明经典 NP 完全性以来,SAT 被广泛作为诸多组合优化与验证问题的统一建模平台。与传统的“为具体问题手写算法”相比,SAT 的优势在于统一性、复用性与工程可行性。\\
本项目面向教学与实践,目标是从零实现一个可用的 DPLL 求解器,覆盖 CNF 解析、内部结构、DPLL 核心流程、分支启发式与性能度量,并将数独问题完整地归约到 SAT 进行求解,构成“理论--实现--应用”的闭环。

从理论谱系上看,SAT 是第一个被证明为 NP 完全的问题,具有“归约枢纽”的地位。Davis--Putnam(DP)消去法与后续 Davis--Logemann--Loveland(DPLL)算法开启了以“系统化搜索 + 逻辑传播”为核心的判定方法:
\begin{itemize}
  \item DP 强调基于分辨率与变量消去,而 DPLL 基于“递归二分 + 单子句传播 + 回溯”的搜索框架;
  \item 现代 CDCL(Conflict-Driven Clause Learning)在 DPLL 基础上加入冲突分析、学习子句、重启策略与相位保存,已成为工业求解的事实标准;
  \item 尽管 CDCL 更强,但 DPLL 仍是理解 SAT 求解器内部机制的“最小充分内核”:传播、分支、回溯、数据结构的相互作用决定了搜索效率的数量级差异。
\end{itemize}

从应用视角看,SAT 的“通用建模”能力极强:
\begin{itemize}
  \item 硬件设计与形式化验证:等价性检查(CEC)、有界模型检测(BMC)、时序属性验证等将电路/状态机转化为可满足性问题;
  \item 软件工程:路径可达性、缺陷定位、程序合成中的候选验证;
  \item 规划与运筹:经典 AI 规划、日程排班、资源分配;
  \item 密码与安全:差分/线性密码分析中的约束求解、协议栈安全属性验证;
  \item 组合设计与解谜:数独、Kakuro、拉丁方、图着色等均可自然归约。
\end{itemize}

以数独为代表的离散谜题提供了一个直观、闭环的示范平台:从约束建模(行/列/宫/对角/窗口)、到 CNF 编码、到 SAT 求解、再到解的还原与可视化,完整串联了“\textbf{问题 \(\to\) 建模 \(\to\) 求解 \(\to\) 验证}”的工程流程。相比直接写一个“特定于数独”的回溯求解器,基于 SAT 的方案更具通用性:一旦编码到 CNF,后续就可以复用更先进的求解技术,无需为每个新问题重新开发搜索引擎。

本项目聚焦 DPLL 的教学实现,意义体现在:
\begin{enumerate}
  \item \textbf{理解本质:}用最小内核呈现 SAT 求解的关键要素(传播/回溯/分支),厘清复杂工程优化背后的“底层语义”;
  \item \textbf{体系方法:}从文件解析到内部存储、从启发式到计时评测,走通一条可复现的工程路径;
  \item \textbf{以点带面:}通过数独应用,示范“SAT 作为统一求解平台”的建模范式,为进一步拓展到 EDA/验证/规划等打基础;
  \item \textbf{实验平台:}为后续替换数据结构(如 watch-lists)、加入学习与重启、切换启发式提供对照基线。
\end{enumerate}

在复杂度层面,DPLL 的最坏时间仍为指数级,但\textbf{数据结构}(如出现表 vs 双 watched literals)、\textbf{启发式}(如 MOMS/DLIS/VSIDS)与\textbf{实现细节}(撤销策略、内存局部性)会显著改变平均性能。基线 DPLL 的价值,不在于“战胜工业级 CDCL”,而在于提供一个可控、可解释、可扩展的试验田。

\subsection{国内外研究现状}
工业级 CDCL 求解器(MiniSAT、Glucose、Kissat、Maple 系列、CaDiCaL 等)采用冲突分析(1-UIP 学习)、双 watched literals、LBD/活动度、重启策略等,已形成成熟工程体系。学术界持续探索更好的启发式、学习子句管理、并行化与领域特定编码。本项目聚焦 DPLL 基础版本,兼顾可读性与可验证性。

\paragraph{发展脉络与关键技术} 经典求解发展可以粗略分段:GRASP 引入系统性的冲突分析;Chaff 推广 VSIDS 与 watched-literals,将传播与启发式的代价显著降低;MiniSAT 用极简实现奠定教学与研究基础;Glucose 引入 LBD 度量指导学习子句管理;Maple/MapleLCMDistAdapt 等对活动度与重启做出新探索;Kissat、CaDiCaL 在实现层面深度打磨(例如缓存友好、分支预测、剔除分支、简化内存布局),进一步提升了稳定性与速度。并行化方面,ManySAT、plingeling 等基于多起点/多策略竞争与子句交换;增量 SAT 与 Assumptions 技术支撑了模型检测、渐进求解等应用场景。预处理/处理中(pre/in-processing)方面,SatELite 及其后续工作系统化了变量消去、子句子集/子句吸收(subsumption/absorption)、阻塞子句消去(BCE)、子句活化(vivification)等流程。

\paragraph{编码与传播优化} 针对结构化约束(如基数/伪布尔/XOR),不同编码在子句规模、传播强度与求解器兼容性间做权衡:
\begin{itemize}
  \item 基数约束:序列(sequential)与多计数器(cardinality networks)编码在规模与传播性上各有优势;
  \item 伪布尔:基于 PB 到 CNF 的传播保持/近似保持编码,或通过混合 SAT+PB 处理;
  \item XOR:高斯消元与 CNF 混合处理可显著提升密码学实例的求解能力。
\end{itemize}

\paragraph{应用与评测} SAT Competition/SAT Race 提供了公平的评测场,追踪了近二十年来的技术演化。应用层面,SAT 已渗透到 EDA、软件验证、规划调度、AI 推理、组合搜索等核心领域。国内高校与研究机构在教学实现、工具迁移与特定领域应用方面也有持续贡献。

\paragraph{本项目定位} 我们选择以 DPLL 为主干,刻意约束工程复杂度,以凸显“传播--分支--回溯”与“数据结构--启发式”之间的因果关系;同时在应用上选择数独,兼顾可视化与可验证性,为后续进一步接入 watch-lists、CDCL 与 GUI 留出空间。

\subsection{课程设计的主要研究工作}
\begin{itemize}
  \item 实现 DIMACS CNF 解析与内部结构;
  \item 设计出现表与变更栈,支持高效传播与撤销;
  \item 实现 DPLL:单子句传播、回溯、分支策略;
  \item 设计两种分支策略并对比耗时,输出优化率;
  \item 实现数独生成、挖洞保唯一、CNF 归约与解还原;
  \item 提供控制台交互与结果落盘。
\end{itemize}

为确保“可复现、可对比、可复用”,我们进一步细化以下工作要点:
\begin{enumerate}
  \item \textbf{解析与验证:}对 DIMACS 解析进行两遍扫描(统计 + 构造),支持 c/p 行、空白与尾随零的鲁棒处理;提供 \texttt{print\_cnf} 逐子句打印,便于人工对照验证;
  \item \textbf{内部结构:}采用“按变量分桶”的正/负极性出现表以降低遍历开销;对子句维持 \texttt{current\_length} 与 \texttt{assignment\_status},支持 O(1) 更新与检查;
  \item \textbf{撤销机制:}以 \texttt{change\_stack}+\texttt{change\_counts[depth][..]} 记录本层操作,按“先 SHRUNK 后 SATISFIED”的顺序精确回滚,避免额外全表扫描;
  \item \textbf{启发式对比:}method=1(近似 MOMS)聚焦最短子句参与度,method=2 为顺序基线;以相同算例对比两者时延并计算优化率;
  \item \textbf{数独编码:}在标准行/列/宫约束基础上,加入一条对角线与两个 3\(\times\)3 窗口的扩展约束(与源码一致),并输出题面单子句;
  \item \textbf{结果格式:}统一输出 \texttt{.res}(s/v/t),其中时间以毫秒计,便于脚本化统计;
  \item \textbf{交互与演示:}提供控制台菜单串联“生成--挖洞--归约--求解--还原--展示/交互”,方便课堂演示与验收;
  \item \textbf{扩展空间:}代码层面预留了切换启发式、替换数据结构与接入 GUI 的接口,便于后续升级到 watch-lists 与 CDCL。
\end{enumerate}

\section{系统需求分析与总体设计}
\subsection{系统需求分析}
\textbf{功能性需求:}
\begin{itemize}
  \item 读取 CNF:支持 c 注释、p 头行、子句行;
  \item 解析验证:可选择打印 CNF 以用于人工对照;
  \item SAT 求解:基于 DPLL,支持两种分支策略;
  \item 时间统计:返回并打印耗时,写入 .res;
  \item 结果保存:生成 ``.res''(s/v/t);
  \item 数独应用:生成题面、归约 CNF、调用求解、解还原、交互编辑;
  \item 菜单交互:生成数独/求解 CNF/展示结果/交互模式。
\end{itemize}
\textbf{非功能性需求:}正确性、可维护、复用性与基本性能。

\paragraph{使用者与场景} 主要面向课程学习者与助教评阅:学习者可通过不同算例观察启发式差异;助教可据 \texttt{.res} 检查结果与时间。典型场景包括:\emph{(i)}读取给定 CNF 并求解;\emph{(ii)}生成数独并通过 SAT 求解验证唯一性;\emph{(iii)}手动交互填写数独并观察解的变化。

\paragraph{接口与输入输出} 输入:CNF 文件路径或菜单选择;输出:控制台日志与 \texttt{.res} 文件。出错处理:文件不存在/解析失败/格式不合法时给出可读性提示;当 p 行与实际子句数不一致时自动以实际计数回写内存结构并警告。

\paragraph{约束与兼容} 目标平台 Windows/MinGW;内存与时间预算以教学算例为主;代码采用 C/C++ 标准库与最小依赖,Qt GUI 为可选项(默认控制台)。

\paragraph{性能与度量} 时间以 \texttt{clock()} 采集;为增强可比性,建议同一算例重复多次取中位数;优化率定义为
\[
  	ext{Optimization}(\%) = \frac{t_{base} - t_{opt}}{t_{base}} \times 100\% \, ,
\]
其中 \(t_{base}\) 为基线策略耗时,\(t_{opt}\) 为优化策略耗时(源码中以 method=2 为基线、method=1 为优化进行对照)。

\paragraph{可测试性与可维护性} 通过 \texttt{print\_cnf} 实现解析验证;\texttt{.res} 输出统一格式便于脚本检查;模块化划分(解析/求解/应用)降低耦合;核心常量集中在 \texttt{definition.h},便于调整规模上限。

\subsection{系统总体设计}
模块划分:I/O 模块、解析模块、DPLL 模块、计时与输出模块、数独模块。\\
流程示意(以 mermaid 伪码呈现):
\begin{lstlisting}[language={}]
flowchart TD
    M[main] -->|菜单| S1[数独生成]
    M -->|菜单| S2[读取CNF并求解]

    S2 --> R(read_cnf_file)
    R --> P(preprocess)
    P --> D(dpll)
    D --> W(write_result)

    S1 --> G(createFullGrid)
    G --> H(digHoles)
    H --> C(writeSudokuToCNF)
    C --> SD(satsolver)
    SD --> B(CnftoSudoku)
    B --> V[printSolvedGrid/交互]
\end{lstlisting}

\paragraph{总体架构与分层} 系统采用“\textbf{界面层(CLI/可选 GUI)}--\textbf{应用层(数独)}--\textbf{求解内核(SAT)}”的三层逻辑:
\begin{itemize}
  \item \textbf{界面层} 提供菜单、参数获取与基础 I/O;
  \item \textbf{应用层} 负责数独的生成、挖洞、归约与还原;
  \item \textbf{求解内核} 负责 CNF 解析、DPLL 搜索、时间统计与结果写回。
\end{itemize}

\paragraph{关键模块职责}
\begin{itemize}
  \item 解析模块(\texttt{read\_cnf\_file}/\texttt{print\_cnf}):两遍扫描构建子句与出现表;
  \item 传播与撤销(\texttt{assign\_value}/\texttt{unassign\_value}):维护子句满足/收缩状态与单子句堆;
  \item 分支(\texttt{select\_branching\_variable}/\texttt{select\_basic\_method}):在性能与实现复杂度之间权衡;
  \item 主过程(\texttt{dpll}/\texttt{satsolver}):组织预处理、递归、计时与结果输出;
  \item 数独(\texttt{createFullGrid}/\texttt{digHoles}/\texttt{writeSudokuToCNF}/\texttt{CnftoSudoku}):串联题面与求解内核。
\end{itemize}

\paragraph{数据与控制流} CNF 管线按照“文件 \(\to\) 解析 \(\to\) 内部结构 \(\to\) DPLL \(\to\) \texttt{.res}”推进;数独管线按照“生成 \(\to\) 挖洞 \(\to\) 归约 \(\to\) SAT \(\to\) 还原/展示”推进。两个管线在 \texttt{satsolver} 处汇合,形成复用。

\paragraph{接口契约(摘选)} 
\begin{itemize}
  \item \texttt{int read\_cnf\_file(char* fn)}:前置 \(fn\) 可读;后置内存结构一致且计数正确;失败返回 0;
  \item \texttt{int dpll()}:在给定内部状态下返回 \(\{\text{SAT},\text{UNSAT}\}\),并保证所有局部赋值得到正确撤销;
  \item \texttt{void write\_result(int sat,double t,char* fn)}:生成 \texttt{.res},其中 \texttt{v} 行长度等于变量数、\texttt{t} 为毫秒;
  \item \texttt{void writeSudokuToCNF(...)}:输出子句头与正文一致,回填 \texttt{p cnf} 的子句计数。
\end{itemize}

\paragraph{可扩展性} 在不改变对外接口的前提下,可以:\emph{(i)}将出现表替换为双 watched literals;\emph{(ii)}引入学习与重启升级为 CDCL;\emph{(iii)}以编译开关切换启发式或添加新的启发式;\emph{(iv)}接入 Qt GUI 展示数独交互与动画传播效果。

\section{系统详细设计}
\subsection{数据结构的定义与关系}
核心结构(摘自 \texttt{definition.h}):Clause(\texttt{literals}、\texttt{assignment\_status}、\texttt{original\_length}、\texttt{current\_length}、\texttt{is\_satisfied}、\texttt{unit\_literal})、LiteralOccurrence/OccurrenceList、ChangeRecord、VariableResult,以及全局的单子句栈、变更栈与分层计数、出现表与标记数组等。\\
设计要点:出现表便于按出现次数更新;\texttt{change\_counts} 按层统计以 O(变更数) 撤销;\texttt{unit\_literal} 有利于撤销时清理 in\_unit 标记。

\subsubsection{SAT/DPLL 核心数据结构(C 代码)}
\begin{lstlisting}[language=C]
#define MAX_VARS     100000
#define MAX_CLAUSES 1000000

/* 子句:DPLL 操作的核心对象 */
typedef struct {
  int *literals;          // 子句中的文字数组
  int *assignment_status; // 每个文字是否已被赋值
  int original_length;    // 初始长度
  int current_length;     // 当前未被赋值的文字数
  int is_satisfied;       // 子句是否已满足
  int unit_literal;       // 若为单子句时的唯一未赋值文字
} Clause;

/* 文字在子句中的一次出现位置 */
typedef struct {
  int clause_index;       // 子句编号
  int literal_position;   // 在该子句中的位置
} LiteralOccurrence;

/* 某变量(正/负极性)的出现表 */
typedef struct {
  LiteralOccurrence *list; // 出现列表
  int count;               // 出现次数
  int capacity;            // 分配容量
} LiteralOccurrenceList;

/* 回溯撤销记录 */
typedef struct { int clause_index, literal_position; } ChangeRecord;

/* 变量赋值结果 */
typedef struct { int value; } VariableResult; // TRUE/FALSE/UNASSIGNED

/* 全局(或模块内全局)状态 */
extern int num_vars, original_formula_length, current_formula_length, max_clause_length;
extern Clause *clauses;

extern LiteralOccurrenceList *pos_literals, *neg_literals; // 正/负极性出现表
extern int *in_unit_pos, *in_unit_neg;                     // 是否在单子句堆

extern VariableResult results[MAX_VARS + 1];

extern int unit_clause_stack[MAX_CLAUSES]; // 单子句堆
extern int unit_stack_size;

extern ChangeRecord change_stack[MAX_CLAUSES * 10]; // 撤销栈
extern int change_stack_top;                         // 撤销栈顶
extern int change_counts[MAX_VARS * 2][2];          // [depth][SATISFIED/SHRUNK]
extern int depth;                                    // 当前搜索深度
\end{lstlisting}

\subsubsection{数独相关数据结构(C 代码)}
\begin{lstlisting}[language=C]
#define SIZE 9
#define BOX_SIZE 3

/* 数独网格与派生网格 */
extern int grid[SIZE][SIZE];        // 当前完整/题面网格
extern int playerGrid[SIZE][SIZE];  // 玩家交互网格(可改动)
extern int entireGrid[SIZE][SIZE];  // 生成的完整解备份
extern int initplaygrid[SIZE][SIZE];// 初始题面备份
extern int solvedGrid[SIZE][SIZE];  // 由 .res 还原出的解

/* 变量映射:X(i,j,k) ↔ (i-1)*81+(j-1)*9+k */
static inline int ChangetoLiteral(int row, int col, int num) {
  return (row-1)*81 + (col-1)*9 + num; // 1-based 参数
}
\end{lstlisting}

\subsubsection{作用与关系说明}
\paragraph{Clause(子句)} 存储每个子句的文字、赋值状态与满足/收缩信息,是传播与回溯的直接载体。\texttt{current\_length} 递减至 1 触发单子句;为 0 且未满足即产生冲突。

\paragraph{LiteralOccurrence/List(出现表)} 对每个变量分别维护正负极性的出现位置集合。传播时:满足含 \(l\) 的子句、收缩含 \(\lnot l\) 的子句——都可通过出现表在 O(出现次数) 内完成,避免全式扫描;启发式统计(最短子句参与度)亦复用该表。

\paragraph{ChangeRecord/Stacks(撤销)} 所有对 \texttt{clauses} 的修改(满足/收缩)都会以 \{c,-1\}/\{c,pos\} 形式压入 \texttt{change\_stack},并在 \texttt{change\_counts[depth][..]} 计数。回溯时按“先 SHRUNK 后 SATISFIED”精确逆操作,恢复上一层一致状态。

\paragraph{VariableResult(变量赋值)} 记录每个变量当前取值,分支与输出都依赖该结构;对局部赋值,在递归退出时须与撤销同步恢复为 UNASSIGNED。

\paragraph{unit\_clause\_stack 与 in\_unit 标记} 维护当前待处理的单子句文字集合,并用 \texttt{in\_unit\_pos/neg} 标记避免重复入堆、快速检测同一变量正/负同时入堆的矛盾,触发回溯。

\paragraph{depth 与 change\_counts} 按搜索层次分组记录变更,配合 LIFO 的 \texttt{change\_stack} 实现 O(变更数) 的回溯撤销,避免对子句与出现表做额外扫描。

\paragraph{数独网格族(grid/playerGrid/entireGrid/initplaygrid/solvedGrid)}
\begin{itemize}
  \item \texttt{grid}:生成完整盘面后挖洞得到题面,后续用于 CNF 归约;
  \item \texttt{playerGrid}:控制台交互写入/清除数字,不影响 SAT 求解流程;
  \item \texttt{entireGrid}:生成时的完整解备份,可用于对照验证;
  \item \texttt{initplaygrid}:题面初始快照,用于重置交互或对比;
  \item \texttt{solvedGrid}:从 \texttt{.res} 的 \texttt{v} 行正文字还原出的最终解。
\end{itemize}

\paragraph{数独 \(\to\) SAT 的数据流} \texttt{grid}(题面)\(\to\) \texttt{writeSudokuToCNF} 生成 CNF(含行/列/宫/对角/窗口与题面单子句)\(\to\) SAT/DPLL 得到 \texttt{.res} \(\to\) \texttt{CnftoSudoku} 将正文字映回 \texttt{solvedGrid} 并展示/校验。

\begin{lstlisting}[language={}]
关系总览(简化)

 CNF(.cnf) --> read_cnf_file --> clauses[] + pos/neg_literals[]
                 |            \
                 |             +--> unit_clause_stack
                 |                  ^      |
                 v                  |      v
              assign/unassign <---- change_stack(+counts)
                 |
                 v
                dpll() ----> results[] ----> .res

 Sudoku: grid --> writeSudokuToCNF --> .cnf --> SAT --> .res --> CnftoSudoku --> solvedGrid
\end{lstlisting}

\subsection{DPLL 算法设计与推导}
\paragraph{算法不变式}
在递归深度 \(d\) 的任何时刻,保持以下不变式:
\begin{itemize}
  \item I1(子句一致性):对每个子句 \(C\),若存在取真文字则 \(C.\texttt{is\_satisfied}=1\) 且 \(C.\texttt{current\_length}\) 可不等于 \(\texttt{original\_length}\);若未满足,则 \(C\) 中被赋值为假或与赋值冲突的文字被标记为已用,\(\texttt{current\_length}\) 等于未赋值文字个数;
  \item I2(单子句队):\texttt{unit\_clause\_stack} 中仅包含“尚未处理”的单子句文字,且 \texttt{in\_unit\_pos/neg} 防止重复入队;
  \item I3(撤销完备):\texttt{change\_stack} 记录了自深度 \(d\) 以来的所有子句状态改变,\texttt{change\_counts[d][..]} 按类型计数;
  \item I4(赋值一致):\texttt{results} 数组与子句/出现表的修改保持一致,且不会同时存在 \(x\) 与 \(\lnot x\) 入单子句队的情况;若出现则立即判冲突。
\end{itemize}

\paragraph{核心流程与伪代码}
\begin{lstlisting}[language=C]
// 传播:将文字 lit 赋真并进行单子句传播,返回是否冲突
bool assign_value(int lit) {
  int var = abs(lit); int isPos = (lit > 0);
  if (results[var].value != 0) return results[var].value == (isPos ? 1 : -1);
  results[var].value = isPos ? 1 : -1;

  // 满足包含 lit 的子句
  LiteralOccurrenceList *satList = isPos ? &pos_literals[var] : &neg_literals[var];
  for (int i = 0; i < satList->count; ++i) {
    int c = satList->list[i].clause_index;
    if (!clauses[c].is_satisfied) {
      clauses[c].is_satisfied = 1;
      change_stack[change_stack_top++] = (ChangeRecord){c, -1};
      change_counts[depth][0]++; // SATISFIED
    }
  }

  // 收缩包含 ~lit 的子句
  LiteralOccurrenceList *shrList = isPos ? &neg_literals[var] : &pos_literals[var];
  for (int i = 0; i < shrList->count; ++i) {
    int c = shrList->list[i].clause_index;
    int p = shrList->list[i].literal_position;
    if (!clauses[c].is_satisfied && !clauses[c].assignment_status[p]) {
      clauses[c].assignment_status[p] = 1;
      clauses[c].current_length--;
      change_stack[change_stack_top++] = (ChangeRecord){c, p};
      change_counts[depth][1]++; // SHRUNK
      if (clauses[c].current_length == 1) {
        // 找到唯一未赋值文字,推入单子句栈
        int u = clauses[c].unit_literal; // 预先缓存或扫描一次
        if (u == -lit) return false;     // 直接冲突
        // 防重加入
        // ... (in_unit_pos/neg 标记检查)
        unit_clause_stack[unit_stack_size++] = u;
      } else if (clauses[c].current_length == 0) {
        return false; // 空子句,冲突
      }
    }
  }
  return true;
}

// 撤销:回溯到上层 depth-1
void unassign_to_prev_depth() {
  // 先撤销 SHRUNK,再撤销 SATISFIED
  for (int i = 0; i < change_counts[depth][1]; ++i) {
    ChangeRecord r = change_stack[--change_stack_top];
    clauses[r.clause_index].assignment_status[r.literal_position] = 0;
    clauses[r.clause_index].current_length++;
  }
  for (int i = 0; i < change_counts[depth][0]; ++i) {
    ChangeRecord r = change_stack[--change_stack_top];
    clauses[r.clause_index].is_satisfied = 0;
  }
  change_counts[depth][0] = change_counts[depth][1] = 0;
  // 清理本层设置的结果与单子句标记,可按需回滚
}

int dpll() {
  // 单子句预处理
  while (unit_stack_size) {
    int lit = unit_clause_stack[--unit_stack_size];
    if (!assign_value(lit)) return 0; // 冲突
  }
  // 检查是否已满足
  if (current_formula_length == 0) return 1;

  // 选择分支变量(见下一小节)
  int branch_lit = select_branching_variable();

  // 尝试正极性
  depth++;
  if (assign_value(branch_lit) && dpll()) return 1;
  unassign_to_prev_depth(); depth--;

  // 尝试负极性
  depth++;
  if (assign_value(-branch_lit) && dpll()) return 1;
  unassign_to_prev_depth(); depth--;
  return 0;
}
\end{lstlisting}

\paragraph{正确性与终止性要点}
传播阶段维护 I1--I4,因此任何产生的空子句都可被正确检测为冲突;回溯通过完整的变更栈逆操作保证“历史可逆”。每步分支严格减少“未赋值变量数”或导致冲突回溯,因此深度有限;同时每层仅二分一次,搜索树有限,算法终止。声音性来源于仅在逻辑蕴含下传播、在冲突时回溯并探索另一分支;完备性源于穷举搜索覆盖了所有赋值。

\subsection{分支启发式与复杂度分析}
\paragraph{MOMS 变体(method=1)}
设当前公式的最短子句长度为 \(L_{min}\)。对每个未赋值变量 \(v\),统计它在所有长度为 \(L_{min}\) 的子句中的正/负出现次数 \(p_c(v), n_c(v)\)。打分函数取
\[
  	ext{score}(v) = (2\,p_c(v)+1)\,(2\,n_c(v)+1)
\]
直觉:同时在许多最短子句中出现、且正负极性都活跃的变量,能更快触发传播或冲突,缩小搜索空间。极性选择可取 \(\text{pol}(v) = \operatorname*{argmax}_{s\in\{+,-\}}\{\#\text{occurrences of } v^s \text{ in } L_{min}\}\)。

\paragraph{基线(method=2)}
简单选择第一个未赋值变量,并优先尝试正极性。该策略实现成本低、可作为可重复的对照基线。

\paragraph{实现摘录(简化)}
\begin{lstlisting}[language=C]
int select_branching_variable() {
  if (method == 2) {
    for (int v = 1; v <= num_vars; ++v)
      if (results[v].value == 0) return v; // 默认正极性
  }
  // method == 1 : 近似 MOMS
  int Lmin = max_clause_length;
  for (int c = 0; c < original_formula_length; ++c)
    if (!clauses[c].is_satisfied) Lmin = (clauses[c].current_length < Lmin) ? clauses[c].current_length : Lmin;
  int bestV = -1; long bestScore = -1; int bestPol = +1;
  for (int v = 1; v <= num_vars; ++v) if (results[v].value == 0) {
    int pc = 0, nc = 0;
    // 遍历出现表,统计仅对长度为 Lmin 的子句
    for (int i = 0; i < pos_literals[v].count; ++i) {
      int c = pos_literals[v].list[i].clause_index;
      if (!clauses[c].is_satisfied && clauses[c].current_length == Lmin) pc++;
    }
    for (int i = 0; i < neg_literals[v].count; ++i) {
      int c = neg_literals[v].list[i].clause_index;
      if (!clauses[c].is_satisfied && clauses[c].current_length == Lmin) nc++;
    }
    long sc = (long)(2*pc+1) * (long)(2*nc+1);
    if (sc > bestScore) { bestScore = sc; bestV = v; bestPol = (pc >= nc) ? +1 : -1; }
  }
  return bestPol * bestV; // 返回带极性的文字
}
\end{lstlisting}

\paragraph{复杂度讨论}
设变量数 \(n\)、子句数 \(m\)、总出现次数 \(T=\sum_C |C|\)。
\begin{itemize}
  \item 传播:一次 \texttt{assign\_value(lit)} 访问 \(\text{occ}(lit) + \text{occ}(\lnot lit)\) 个出现,故均摊为 \(O(\text{occ}(lit)+\text{occ}(\lnot lit))\)。每个出现至多被“收缩一次”,因而总传播工作 \(O(T)\) 封顶(不计回溯重做);
  \item 分支选择:method=2 为 \(O(n)\);method=1 需先求 \(L_{min}\)(\(O(m)\))并按出现表统计最短子句出现数(总体 \(\le O(T)\),但通常远小于 \(T\));
  \item 空间:子句数组与出现表为主,约 \(O(T)\);撤销栈与单子句栈为 \(O(T)\) 级别上界;
  \item 最坏时间:仍为指数级,具体由分支顺序决定;启发式好坏直接影响搜索树大小与传播密度。
\end{itemize}

\subsection{数独生成与归约设计}
\paragraph{Las Vegas 生成与回溯}
  exttt{createFullGridLasVegas} 先随机打乱 1..9 的顺序,对若干格进行随机一致性填数,随后用回溯解完整盘面;若在限定时间窗(如 1s)内失败,则重启随机序列。\texttt{createFullGrid} 根据尝试失败次数自适应调整预填个数,使得整体期望时间更平稳。

\paragraph{挖洞保唯一}
  exttt{digHoles} 随机选择已填格置空,调用 \texttt{hasonly} 进行唯一性检查:\texttt{countSolutions} 使用带两个解上界的 DFS(找到两解即可提前返回)保证判定效率。若唯一性被破坏,则回滚该挖洞操作。重复直到达到目标空格数或无可挖为止。

\paragraph{CNF 编码条款}
变量映射:\(X(i,j,k) \Leftrightarrow (i{-}1)\cdot 81 + (j{-}1)\cdot 9 + k\),其中 \(i,j,k\in\{1..9\}\)。核心约束:
\begin{enumerate}
  \item 每格恰一:对每个 \((i,j)\),\(\bigvee_k X(i,j,k)\) 与 \(\forall k_1<k_2,\; \lnot X(i,j,k_1)\lor\lnot X(i,j,k_2)\);
  \item 每行每数至少一:对每个 \((i,k)\),\(\bigvee_j X(i,j,k)\);
  \item 每列每数至少一:对每个 \((j,k)\),\(\bigvee_i X(i,j,k)\);
  \item 每宫每数恰一:3\(\times\)3 宫内对固定 \(k\) 做 OR 与两两互斥;
  \item 扩展约束:主对角线唯一性、两个 3\(\times\)3 窗口的附加互斥(与源码一致);
  \item 题面:若 \(grid[i][j]=k\),加入单子句 \(X(i,j,k)\)。
\end{enumerate}

\paragraph{生成关键代码片段(简化)}
\begin{lstlisting}[language=C]
// 每格至少一
for (int i=1;i<=9;++i) for (int j=1;j<=9;++j) {
  // OR 子句
  for (int k=1;k<=9;++k) fprintf(out, " %d", ChangetoLiteral(i,j,k));
  fprintf(out, " 0\n");
  // 两两互斥
  for (int k1=1;k1<=9;++k1) for (int k2=k1+1;k2<=9;++k2)
    fprintf(out, " -%d -%d 0\n", ChangetoLiteral(i,j,k1), ChangetoLiteral(i,j,k2));
}

// 行至少一
for (int i=1;i<=9;++i) for (int k=1;k<=9;++k) {
  for (int j=1;j<=9;++j) fprintf(out, " %d", ChangetoLiteral(i,j,k));
  fprintf(out, " 0\n");
}

// 列至少一
for (int j=1;j<=9;++j) for (int k=1;k<=9;++k) {
  for (int i=1;i<=9;++i) fprintf(out, " %d", ChangetoLiteral(i,j,k));
  fprintf(out, " 0\n");
}

// 宫内互斥(示例)
for (int bi=0; bi<3; ++bi) for (int bj=0; bj<3; ++bj)
  for (int k=1;k<=9;++k) {
    for (int p=0;p<9;++p) for (int q=p+1;q<9;++q) {
      int i1=bi*3 + p/3 + 1, j1=bj*3 + p%3 + 1;
      int i2=bi*3 + q/3 + 1, j2=bj*3 + q%3 + 1;
      fprintf(out, " -%d -%d 0\n", ChangetoLiteral(i1,j1,k), ChangetoLiteral(i2,j2,k));
    }
  }
\end{lstlisting}

\paragraph{子句规模估算}
标准 9\(\times\)9 数独下:每格恰一产生 9 OR + C(9,2)=36 个互斥子句,共 \(81\times(1+36)=2997\) 条;行/列至少一各 \(9\times9=81\) 条;每宫互斥每 \(k\) 有 C(9,2)=36 条,共 \(9\times9\times 36=2916\) 条。加上扩展约束与题面单子句,整体在几千到一万余子句量级,易于 DPLL 处理。

\paragraph{唯一性校验(伪代码)}
\begin{lstlisting}[language=C]
int solutions = 0;
bool dfs_cell(int idx) {
  if (idx == 81) { solutions++; return solutions < 2; }
  int i = idx / 9, j = idx % 9;
  if (grid[i][j] != 0) return dfs_cell(idx+1);
  for (int k=1;k<=9;++k) if (safe(i,j,k)) {
    grid[i][j]=k;
    if (!dfs_cell(idx+1)) return false; // 找到两解即剪枝
    grid[i][j]=0;
  }
  return true;
}
bool hasonly() { solutions=0; dfs_cell(0); return solutions==1; }
\end{lstlisting}

\subsection{基于 Qt 的 GUI 设计}
\paragraph{界面结构}
GUI 采用 Qt Widgets:主窗口包含 9\(\times\)9 网格(\texttt{QTableWidget} 或自绘)、“生成题面”“求解”“清空/重置”等按钮,以及状态栏显示耗时与结果。题面编辑可限制为 1..9 的输入并高亮冲突。

\paragraph{信号与槽}
按钮点击触发槽函数:\texttt{onGenerate()} 生成并渲染题面;\texttt{onSolve()} 将当前网格写入临时 CNF,在线程中调用 SAT 求解,完成后读取 \texttt{.res} 更新 UI;\texttt{onClear()} 重置为初始题面。

\paragraph{线程化求解}
为避免阻塞 UI 线程,使用 \texttt{QThread} 或 \texttt{QtConcurrent::run} 将求解放入工作线程,通过信号在完成时更新界面。

\paragraph{示例代码(简化)}
\begin{lstlisting}[language=C++]
// Worker: 在后台线程执行 SAT 求解
class SolverWorker : public QObject {
  Q_OBJECT
public slots:
  void solveCNF(QString cnfPath) {
    auto t0 = std::chrono::steady_clock::now();
    int sat = satsolver(cnfPath.toLocal8Bit().data()); // 复用已有 C 接口
    auto t1 = std::chrono::steady_clock::now();
    emit finished(sat, std::chrono::duration_cast<std::chrono::milliseconds>(t1-t0).count());
  }
signals:
  void finished(int sat, qint64 ms);
};

// MainWindow 片段
void MainWindow::onSolve() {
  // 1) 将当前题面写入临时 CNF 文件
  QString cnf = QStandardPaths::writableLocation(QStandardPaths::TempLocation) + "/sudoku.cnf";
  writeSudokuToCNF(cnf.toLocal8Bit().data());

  // 2) 线程化执行
  QThread *th = new QThread(this);
  SolverWorker *worker = new SolverWorker();
  worker->moveToThread(th);
  connect(th, &QThread::started, [=]{ worker->solveCNF(cnf); });
  connect(worker, &SolverWorker::finished, this, [=](int sat, qint64 ms){
    th->quit(); th->wait(); worker->deleteLater(); th->deleteLater();
    statusBar()->showMessage(QString("结果: %1, 用时 %2 ms").arg(sat?"SAT":"UNSAT").arg(ms));
    if (sat) { CnftoSudoku("sudoku.res"); renderGrid(solvedGrid); }
  });
  th->start();
}
\end{lstlisting}

\paragraph{数据绑定与渲染}
将内部 \texttt{grid/playerGrid/solvedGrid} 与表格控件互相映射,输入验证(1..9 或空)在 \texttt{QItemDelegate} 层处理;高亮当前行/列/宫与冲突单元以提升可视性。

\section{系统实现与测试}
\subsection{实现环境与工程结构}
\paragraph{开发环境}
\begin{itemize}
  \item 操作系统:Windows 10/11 x64
  \item 编译工具链:MinGW-w64(工作区含 \texttt{build/} 产物与 \texttt{gcc.exe} 任务)
  \item IDE:Visual Studio Code(已配置 C/C++ 插件与编译任务)
  \item 依赖:标准 C/C++ 运行时;可选 Qt 6.x(GUI)
\end{itemize}

\paragraph{源码布局}
核心源码位于 \texttt{Qt/} 子目录:
\begin{itemize}
  \item SAT 内核:\texttt{definition.h}、\texttt{satsolver.hpp}
  \item 数独应用:\texttt{soudu.hpp}、\texttt{sudutocnf.hpp}、\texttt{cnftosudoku.hpp}
  \item 入口与交互:\texttt{main.cpp}、可选 GUI(\texttt{mainwindow.*})
\end{itemize}

\paragraph{构建与运行}
\begin{enumerate}
  \item 非 Qt 场景:直接使用 VS Code 的“C/C++: gcc 生成活动文件”任务编译当前源文件(例如 \texttt{main.cpp})。
  \item CMake 场景:工作区已含 \texttt{build/} 与 \texttt{CMakeFiles/},可通过 CMake 预设或直接构建目标(如 \texttt{sudoku.exe})。
  \item 运行:可执行文件位于工作区根目录或 \texttt{build/};数独 CNF/RES 样例在根目录与 \texttt{Qt/} 下均有备份。
\end{enumerate}

\subsection{核心接口与调用关系}
\paragraph{SAT 内核(选摘)}
\begin{itemize}
  \item \texttt{int read\_cnf\_file(char* fn)}:读取 DIMACS \texttt{.cnf},构建子句与出现表
  \item \texttt{int dpll()}:执行 DPLL 搜索并返回 \{0: UNSAT, 1: SAT\}
  \item \texttt{int select\_branching\_variable()}:按 method 选择分支变量/极性
  \item \texttt{void write\_result(int sat, double ms, char* fn)}:生成 \texttt{.res}
  \item \texttt{void free\_memory()}:释放构建的动态内存
\end{itemize}

\paragraph{数独应用(选摘)}
\begin{itemize}
  \item \texttt{void createFullGrid()} / \texttt{createFullGridLasVegas()}:生成完整盘面
  \item \texttt{void digHoles(int holes)}:挖洞并保证唯一解
  \item \texttt{void writeSudokuToCNF(char* out)}:将题面编码为 CNF 文件
  \item \texttt{void CnftoSudoku(char* res)}:解析 \texttt{.res} 还原解盘
\end{itemize}

\paragraph{调用关系要点}
统一通过 \texttt{satsolver(...)} 串联“读 CNF → DPLL → 写 RES”;数独链路“题面 → writeSudokuToCNF → satsolver → CnftoSudoku → 展示”。

\subsection{测试方案与用例设计}
为保证结果可复现与便于批量评测,测试按“解析 → 功能 → 稳定性 → 性能”四层推进。

\paragraph{解析正确性}
\begin{itemize}
  \item 使用 \texttt{print\_cnf} 打印内部子句,与原 \texttt{.cnf} 行对照(允许空白与注释差异)。
  \item 极端行检测:空子句、重复文字、尾随零、p 头与实际计数不一致时的容错与告警。
\end{itemize}

\paragraph{功能正确性}
\begin{itemize}
  \item SAT 样例:\texttt{1.cnf}、\texttt{6.cnf} 应返回 SAT,\texttt{.res} 的 \texttt{s} 行为 SAT 且 \texttt{v} 行能满足所有子句。
  \item UNSAT 样例:\texttt{11(unsatisfied).cnf} 应返回 UNSAT,\texttt{s} 行为 UNSAT。
  \item 数独链路:生成 → 挖洞 → 归约 → 求解 → 还原,最终 \texttt{solvedGrid} 满足原题面且约束一致。
\end{itemize}

\paragraph{稳定性与边界}
\begin{itemize}
  \item 边界规模:接近 \texttt{MAX\_VARS}/\texttt{MAX\_CLAUSES} 的算例应给出友好提示或正确失败处理;
  \item 零/负文字、越界变量、非 DIMACS 格式:应拒绝并输出出错位置;
  \item 释放内存:重复多次调用 \texttt{satsolver} 不出现泄漏(可在调试器/外部工具下观察)。
\end{itemize}

\paragraph{性能评测(启发式对比)}
在同一算例上,分别以 method=2(基线)与 method=1(优化)运行多次(如 5–10 次),记录毫秒时间,取中位数以降低抖动。

\subsubsection*{测试矩阵(示例)}
\begin{longtable}{p{3cm}p{2cm}p{2cm}p{2cm}p{2cm}p{3cm}}
	oprule
算例 & 变量数 & 子句数 & 基线中位(ms) & 优化中位(ms) & 备注 \\
\midrule
1.cnf &  &  &  &  & 小规模可满足 \\
6.cnf &  &  &  &  & 中规模可满足 \\
11(unsatisfied).cnf &  &  &  &  & 不可满足 \\
数独生成样例 &  &  &  &  & 归约后求解 \\
\bottomrule
\end{longtable}

\subsubsection*{RES 格式与校验}
\begin{itemize}
  \item \texttt{s} 行:\texttt{s SATISFIABLE} 或 \texttt{s UNSATISFIABLE}
  \item \texttt{v} 行:以空格分隔的赋值,正数代表取真,负数代表取假;
  \item \texttt{t} 行:\texttt{t <milliseconds>}
\end{itemize}
校验:读取 \texttt{v} 行构造真值表,逐子句验证至少一个文字为真(SAT 情况)。

\subsection{性能评测与优化率}
\paragraph{统计口径}
	exttt{satsolver} 使用 \texttt{clock()} 计时(毫秒),并写入 \texttt{.res} 的 \texttt{t} 行。多次运行后取中位数作为代表值。

\paragraph{优化率定义}
\[
  	ext{Optimization}(\%) = \frac{t_{base} - t_{opt}}{t_{base}} \times 100\% \, ,
\]
其中 \(t_{base}\) 为 method=2(基线)中位耗时,\(t_{opt}\) 为 method=1(优化)中位耗时。

\paragraph{记录模板(填写即用)}
\begin{longtable}{p{3.2cm}p{2.2cm}p{2.2cm}p{2.2cm}p{2.2cm}p{2.2cm}}
	oprule
算例 & 重复次数 & 基线中位(ms) & 优化中位(ms) & 优化率(\%) & 备注 \\
\midrule
1.cnf & 5 &  &  &  &  \\
6.cnf & 5 &  &  &  &  \\
11(unsatisfied).cnf & 5 &  &  &  &  \\
数独样例 & 5 &  &  &  &  \\
\bottomrule
\end{longtable}

\paragraph{PowerShell 批量评测脚本(可选)}
在 Windows PowerShell 下,可以用如下脚本对一批 \texttt{.cnf} 文件分别以两种 method 运行 N 次并收集中位数(示例展示思路,按需改写可执行路径与参数方式)。

\begin{lstlisting}[language=powershell]
$files = @('1.cnf','6.cnf','11(unsatisfied).cnf','sudoku.cnf')
$runs = 7  # 奇数便于取中位数

function Median($arr){ $s=$arr | Sort-Object; return $s[[Math]::Floor($s.Count/2)] }

foreach($f in $files){
  Write-Host "Testing $f"
  $tBase = @(); $tOpt = @();

  for($i=0; $i -lt $runs; $i++){
    # 运行基线 method=2
    .\main.exe --solve --file $f --method 2 | Out-Null
    $res = Get-Content ($f -replace '.cnf$','.res') | Where-Object { $_ -like 't *' }
    $ms = [int]($res -replace 't ',''); $tBase += $ms

    # 运行优化 method=1
    .\main.exe --solve --file $f --method 1 | Out-Null
    $res = Get-Content ($f -replace '.cnf$','.res') | Where-Object { $_ -like 't *' }
    $ms = [int]($res -replace 't ',''); $tOpt += $ms
  }

  $mb = Median $tBase; $mo = Median $tOpt
  $ratio = if($mb -gt 0){ [math]::Round((($mb-$mo)/$mb)*100,2) } else { 0 }
  Write-Host ("{0} | base={1}ms opt={2}ms | optRate={3}%" -f $f,$mb,$mo,$ratio)
}
\end{lstlisting}

说明:若你的程序当前通过交互菜单选择 method,可临时增加命令行参数或用输入重定向实现批测;上表模板可直接将脚本输出粘贴后填写。

\section{总结与展望}
\subsection{全文总结}
本项目围绕“可复现的教学级 SAT/DPLL 内核 + 数独应用链路”达成以下目标:
\begin{itemize}
  \item \textbf{完整功能闭环}:实现 DIMACS 解析 → 内部结构构建 → DPLL 搜索(传播/分支/回溯)→ 结果写回 RES;数独侧实现“生成/挖洞/归约/求解/还原”。
  \item \textbf{关键数据结构}:以“按变量分桶”的正/负出现表减少无关遍历;以“变更栈 + 分层计数”实现 O(变更数) 回溯;单子句堆配合 in\_unit 标记避免重复与误入。
  \item \textbf{启发式对比}:提供基线(method=2)与近似 MOMS(method=1);在统一脚本与模板下进行中位数对比,能直观看到最短子句驱动的搜索剪枝效果。
  \item \textbf{工程与健壮性}:对 p 头/注释/空白/尾零的宽容解析;在冲突检测、撤销顺序(先 SHRUNK 后 SATISFIED)与结果一致性上给出清晰不变式;多次求解后释放内存,支持批测。
  \item \textbf{GUI 拓展基础}:给出 Qt 线程化求解的示例,复用既有 C 接口,完成从 UI 到 SAT 的数据链路打通。
\end{itemize}

从实验结果(模板与脚本)看,method=1 在多数结构化算例与数独归约上能取得正向优化率;但在部分极小规模或随机噪声占比高的实例上,两者差距可能不显著,这与启发式统计成本与传播获益的权衡一致。整体上,当前实现为进一步演进(watch-lists、CDCL)提供了清晰的基线与对照。

已知限制:
\begin{itemize}
  \item 未采用双 watched literals,传播在“长子句/高出现度”实例上开销偏大;
  \item 未做冲突分析/学习子句/重启,遇到 UNSAT 硬例时回溯树可能极深;
  \item CNF 编码以直观互斥为主,未引入更紧的传播保持编码(如序列/网络编码)。
\end{itemize}

\subsection{工作展望}
结合当前代码结构,给出三条演进路线:
\paragraph{算法与数据结构}
\begin{itemize}
  \item \textbf{Watched-literals}:将出现表替换/并存为双监视指针,传播从“扫描整子句”降为“移动监视”事件,显著降低均摊成本;
  \item \textbf{CDCL 增强}:加入冲突分析(1-UIP)、学习子句、LBD 度量与活动度(VSIDS/CHB),配合重启与相位保存;
  \item \textbf{Pre/In-processing}:子句吸收、子句/变量消去、BCE、vivification;对数独类结构化约束引入更紧的编码;
  \item \textbf{启发式库}:在 method 框架下扩展 DLIS、Jeroslow–Wang、活动度驱动的极性与随机扰动,并以脚本化基准观察差异。
\end{itemize}

\paragraph{工程与测量}
\begin{itemize}
  \item \textbf{精细计时与火焰图}:细分读入/预处理/传播/选择/回溯耗时,结合采样分析定位热点;
  \item \textbf{日志与可视化}:可选的逐层搜索日志、冲突路径与回溯深度分布,便于课堂演示;
  \item \textbf{测试与 CI}:添加单元测试(解析/传播/撤销/极小 CNF)、一致性检查脚本;在多平台自动化运行基准;
  \item \textbf{配置化与参数化}:以命令行参数选择启发式、限时/限节点、随机种子与日志级别,便于可重复实验。
\end{itemize}

\paragraph{应用与 GUI}
\begin{itemize}
  \item \textbf{GUI 强化}:网格高亮、冲突提示、步进传播(动画演示)、统计面板(节点数、传播数、回溯深度直方图);
  \item \textbf{更多谜题/模型}:Killer/对角/不等式数独、拉丁方、数和谜(Kakuro);亦可尝试图着色、N 皇后等标准归约;
  \item \textbf{示教模式}:提供“讲解模式”,逐步展示单子句、分支决策与冲突回溯的因果链。
\end{itemize}

\section{体会}
结合本次实现与调试过程,有如下经验:
\begin{itemize}
  \item \textbf{数据结构优先}:出现表/监视列表、撤销栈的设计优先级高于“写对功能”,合理的数据结构能将复杂度从“看似可用”降到“可批量评测”;
  \item \textbf{不变式与撤销顺序}:先 SHRUNK 后 SATISFIED 的回滚顺序来自于修改依赖关系;在实现中显式维护不变式(I1–I4)能减少隐性 Bug;
  \item \textbf{启发式的双刃性}:近似 MOMS 对结构化实例受益明显,但在小规模或低结构实例上收益不稳定,测量与分层记录比“感觉”更重要;
  \item \textbf{建模即证明}:将数独严格编码为 CNF 的过程,倒逼我们检查“至少一/至多一/恰一”的覆盖面与边界,从而反向验证了求解器输出的可靠性;
  \item \textbf{可复现的测量}:统一 RES 的 t 行、批量脚本与中位数统计,使对比具备“可复制粘贴”的可信度;
  \item \textbf{边界与健壮}:对 DIMACS 的鲁棒解析、重复求解的内存释放、对异常输入的友好失败,都是把“小实验”做成“可演示项目”的关键。
\end{itemize}

\section*{参考文献}
同“任务书”部分所列。

\section*{附录:样例、截图与脚本}
建议附上 \texttt{print\_cnf} 输出片段与原 CNF 对照、\texttt{.res} 示例(SAT/UNSAT)、数独题面与解的截图,以及不同策略在若干算例上的耗时对比表。

\end{document}
